\documentclass{article}
\usepackage[utf8]{inputenc}
\usepackage[margin=0.75in]{geometry}
\usepackage{amsmath}
\usepackage{natbib}
\usepackage{hyperref}

\begin{document}
	\begin{center}
		\LARGE{\textbf{STA380 Project Proposal}} \\
        \vspace{1em}
        \Large{VaR calculator for Stock Portfolio through Monte Carlo and Geometric Brownian Motion} \\
        \vspace{1em}
        \normalsize\textbf{Jeffrey Huang, Sun Jiawei, Mahan Hooshmandkhayat} \\
        \normalsize{jjeffrey.huang@mail.utoronto.ca}\\
        \normalsize{faye.sun@mail.utoronto.ca}\\
        \normalsize{mahan.hooshmandkhayat@mail.utoronto.ca}\\
        \vspace{1em}
        \normalsize{University of Toronto}
     
	\end{center}
    \begin{normalsize}
        \section{Project Topic}
        Using Monte Carlo and Geometric Brownian Motion to estimate the most realistic value of the risk of a chosen stock portfolio.
        
        \section{Introduction}
        By having a stock portfolio consisting of multiple different stocks from a variety of different sectors, a Value-at-Risk value is used to help us quantify the amount of risk of the entire portfolio. We aim to combine the simulation technique, Monte Carlo, and Geometric Brownian Motion to help us simulate and process our data at a given confidence level into a single quantity that represents the maximum expected loss over time.\\
        
        In this project, the user is provided with a capital of \$1,000,000 and they will be using it to invest in the stock indexes provided in order to create a portfolio of their own. Through their selections, our program will display a final value of the approximate risk that the user's portfolio will have as well as the profits/losses they gain. 

        \section{Simulation vs. Dataset}
        The data will be gathered from using the \textbf{yfR} package that allows us to work with stock data from Yahoo Finance.
        
        \section{Project Details}
        Separating the details into 4 major sections
        \begin{itemize}
        
            \item \textbf{Regime-switching Volatility Predictor:}\\
            The volatility of each specific stock will be found by fitting a Hidden Markov Model to identify its current regime (High, Medium, Low), and then selecting the volatility with respect to its current regime. These regime-specific volatilities, along with the correlation matrix estimated from historical returns, are used to construct the covariance matrix: $$\Sigma_{ij} = \rho_{ij} \cdot \sigma_i \cdot \sigma_j$$
            where $\rho_{ij}$ is the correlation between stocks $i$ and $j$, and $\sigma_i$, $\sigma_j$ are their regime-specific volatilities. We then apply Cholesky factorization to decompose this covariance matrix into a lower triangular matrix $L$ such that: $$\Sigma = LL^T$$ This allows us to transform a vector of independent standard normal random variables $Z$ into correlated random shocks $\tilde{Z}$:
            $$\tilde{Z} = LZ$$ These correlated shocks are then used in the Geometric Brownian Motion simulation to generate realistic portfolio returns.
            
            \item \textbf{Geometric Brownian Motion:} \\
            We will be modeling with a lognormal GBM, $$r_i = \left(\mu_i - \frac{1}{2}\sigma_i^2\right)T + \sigma_i \sqrt{T} \cdot \tilde{Z}_i$$ where $\tilde{Z} \sim N(0, \Sigma)$ is drawn from a multivariate normal distribution with covariance matrix $\Sigma$. $\mu_i$ is the drift or expected return, $\sigma_i$ is the regime-specific volatility from our HMM, $T$ is the time horizon, and $\tilde{Z}_i$ is the correlated random shock obtained from Cholesky decomposition.
            
            \item \textbf{Monte Carlo:} \\
            We aim to run around 10,000 simulations. Each simulation will include the volatility value from our predictor being substituted into our GBM equation. In each simulation, we will receive different values as each change in the ``shock" will affect the final value of our simulation. These values will then be used to calculate the portfolio's Profit/Loss with the portfolio weights. After all simulations are done, we can choose, say, a 95\% confidence interval to represent maximum expected loss, and the remaining 5\% would be the VaR value. 

            \item \textbf{Back testing:} \\
            We will be using Kupiec's test to test 250 days(standard full trading year) in order to test the validity of our VaR. The Kupiec test is used to test the proportion of failures to see how much of it is in accordance with our confidence interval. The main formula used is the Likelihood Ratio test: $$LR_{POF} = -2 \ln \left[ \frac{(1-p)^{T-x} \cdot p^x}{\left(1-\frac{x}{T}\right)^{T-x} \cdot \left(\frac{x}{T}\right)^x} \right]$$
            where $p$ = expected failure rate (e.g., 0.05 for 95\% VaR), $T$ = number of observations in backtest period, $x$ = number of observed violations

        \end{itemize}
        
        \section{User Inputs (Shiny Components)}
        \begin{itemize}
            \item Create a custom stock index portfolio with a certain number of indexes available.
            \begin{itemize}
                \item Tech: Apple, NVIDIA, Tesla
                \item Financial: Berkshire Hathaway, Visa, JP Morgan Chase
                \item Defenses: Lockheed Martin, Northrop Grumman
                \item Consumer Staples: Walmart, Costco
            \end{itemize}
            \item Change the maximum expected loss.
            \begin{itemize}
                \item A movable scale that spans a minimum of 90\% to 99\%.
            \end{itemize}
            \item Change the number of simulations for Monte Carlo.
            \begin{itemize}
                \item A movable scale that allows user to select number of simulations in terms of 1,000 with a minimum of 5,000 and a maximum of 10,000.
            \end{itemize}
            \item Graph display of all the Monte Carlo simulations.
            \item A chart that displays the volatility of stocks, the regime of the stocks, and the volatility of the portfolio.
            \item Display the correlation charts of the portfolio.
            \item Choosing VaR time horizons(\textit{T}) 
            \begin{itemize}
                \item A movable scale that spans a minimum of 1 day to a maximum of 30 days.
            \end{itemize}
        \end{itemize}
        The usage of AI was specifically targeted into learning and gaining a deeper understanding for Cholesky factorization, and the updated symbols from the GBM formula. 

\end{normalsize}
\newpage
\nocite{*}
\bibliographystyle{abbrvnat}
\bibliography{bibliography}
\vspace{0.8cm}


\end{document}